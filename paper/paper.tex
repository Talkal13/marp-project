\documentclass{article}
\usepackage{mathtools}
\usepackage{csquotes}
\usepackage{graphicx}
\usepackage{amsfonts}
\usepackage{amsmath}
\usepackage{listings}

\DeclarePairedDelimiter\abs{\lvert}{\rvert}%
\title{Practica de Métodos Algorítmicos de Resolución de Problemas}
\author{Pablo Vazquez Gomis  \\
	Universidad Complutense de Madrid \\
	}

\date{\today}
\begin{document}

\maketitle


\begin{abstract}
Practica opcional del segundo cuatrimestre para la asignatura de Métodos Algorítmicos de Resolución de Problemas.
\end{abstract}

\section{Introducción}
\subsection{Descripción de la práctica}
\paragraph{Programar el problema de encontrar el clique de tamaño máximo en un grafo no dirigido} 
Dado un grafo $G = (V, E)$, un clique es un conjunto $C \subseteq V$ tal que $\forall u, v \in C, \exists (u, v) \in E$.
\paragraph{}El problema consiste en encontrar el clique de tamaño maximo en $G$, para ello diferentes estrategias pueden ser utilizadas.
La más común es utilizar la técnica de ramificacion y poda. Este problema a sido estudiado por varios investigadores a lo largo de los años.
\paragraph{}En 1990 Carraghan y Pardalos \cite{car} asentaron las bases sobre las que otros investigadoes han ido construyendo. Especialmente introdujeron el concepto del
set $P(C) \subseteq V$, cuya definición informal se reduce a el conjunto de vertices que pueden formar cliqué con $C$ aunque no garantiza $C + P(C)$ sea un clique. 
Es decir $\forall v \in P(C) : C + v$ es cliqué.
\paragraph{} Las estrategias posteriores, en general, tratan de acercar lo maximo posible a la solucion el set $P(V)$ para mejorar la poda.
Una aproximación comun es utilizar el coloreado de grafos para mejorar el upper bound como hacen Tomita y Seki \cite{tom} o J. Konc, D. Janežič \cite{jonc}. 
\subsection{Implementación}

\paragraph{Grafo} El grafo se ha implementado mediante listas de adjacencia ordenadas, para evitar duplicar la información (dado que $(u, v) = (v, u)$ $\forall u, v \in V$)
solo contiene la arista el vertice menor.

Debido a esta implementación el grafo necesita utilizar una lista ordenada, dado que los primeros vertices contienen mas informacion que los últimos.
\section{El algoritmo}
El algoritmo usado es el de ramificacion y poda visto en clase, con una cola de prioridad para explorar los nodos del arbol más prometedores, un test de factibilidad y dos cotas una optimista y otra pesimista para poder podar el arbol.
\subsection{El arbol}
El arbol calcula todas las posibilidades, es decir toda la combinación de nodos $\mathbb{P}(V)$ de menor a mayor tamaño, es decir, en el primer nivel calculamos los sets $C = \{v\}$ $\forall v \in V$. 
En cada nivel generamos los hijos de cada nodo mediante el set $P(V)$ el cual nos garantiza que todos los hijos son solución al problema. 
\paragraph{} Debido a que solo el nodo menor contiene la arista en su lista de adjacencia, se garantiza que ningún set esta repetido en otras ramas.
\subsection{Test de factibilidad}
El test de factibilidad es sencillo, con comprobar que C es solución basta, dado que si C es solucion cualquier $C^\prime \subseteq C$ es solución.
\paragraph{} Para comprobar que un set es solución solo tenemos que hacer una sencilla operación matematica 
\[ \abs{E^\prime} = \dfrac{\abs{C} * (\abs{C} - 1)}{2} \]
\subsection{Cotas}
\subsubsection{Simple}
\paragraph{Optimista.} La cota optimista simple es sencilla de ver utilizando la notación de Carraghan y Pardalos, dado que para cualquier posible solución $C$ optimisticamente todos los elementos en $P(V)$ formaran la solución final. Es decir:
\[ \text{cota-optimista}(C) = \abs{C} + \abs{P(C)} \]
\paragraph{Pesimista.} En este problema es imposible comprobar si explorar los nodos de $P(V)$ si pueden o no formar parte del cliqué maximo por lo que la cota pesimista ha de ser
\[ \text{cota-pesimista}(C) = \abs{C} \]
\subsubsection{Compleja}
\paragraph{Optimista.} La cota optimista compleja se aprovecha de la tecnica del coloreado de vertices presentada por Tomita y Seki \cite{tom} de forma simplificada, damos a un color diferente a todos aquellos vertices $u, v \in P(C)$ tal que $(u, v) \in E^\prime$.
Podemos utilizar esto para delimitar mejor la cota optimista de la siguiente forma:
\[ \text{cota-optimista}(C) = \abs{C} + S\]
Donde $S$ es el numero de colores diferentes.
\paragraph{Pesimista.} La cota pesimista sigue siendo igual que en la Simple por la misma razón.  
\section{Resultados Experimentales}
\subsection{Grafos Aleatorios}
Todas las mediciones se han obtenido aleatoriamente con una semilla "-s 1" para poder replicar los resultados
\newline\newline
\hspace*{-4.2cm}
	\begin{tabular}{| c | c | c c c| c c c | c c c |}
		\hline
		Tamaño & Densidad & Sin Poda & Nodos & Sec/Nodo & Poda Simple & Nodos & Sec/Nodo & Poda Compleja & Nodos & Sec/Nodo \\
		\hline\hline
		10 & 0.1 & 0.0027 & 11 & 0.0001 & 0.0028 & 2 & 0.0006 & 0.0019 & 1 & 0.0004 \\
		\hline
		10 & 0.3 & 0.0927 & 102 & 0.0007 & 0.0462 & 13 & 0.0031 & 0.0334 & 7 & 0.0040 \\
		\hline
		10 & 0.5 & 0.3108 & 795 & 0.0003 & 0.0968 & 39 & 0.0022 & 0.0282 & 17 & 0.0014 \\
		\hline
		10 & 0.7 & 65.574 & 139373 & 0.0003 & 3.4457 & 1958 & 0.0016 & 0.0808 & 35 & 0.0021 \\
		\hline
		10 & 0.9 & 123.5 & 235274 & 0.0004 & 18.874 & 13701 & 0.0012 & 0.1055 & 36 & 0.0026 \\
		\hline
		\hline
		20 & 0.1 & 0.0529 & 61 & 0.0006 & 0.0409 & 7 & 0.0036 & 0.0303 & 4 & 0.0037 \\
		\hline
		20 & 0.3 & 2.0066 & 2269 & 0.0008 & 0.5331 & 127 & 0.0040 & 0.1804 & 21 & 0.0073 \\
		\hline
		20 & 0.5 & - & - & - & - & - & - & 18.679 & 1779 & 0.0103 \\
		\hline
		\hline
		50 & 0.1 & 0.7174 & 536 & 0.0011 & 0.4065 & 57 & 0.0055 & 0.1860 & 12 & 0.0079 \\
		\hline
		50 & 0.3 & - & - & - & 152.62 & 7612 & 0.0199 & 5.6021 & 187 & 0.0290 \\
		\hline \hline
		100 & 0.1 & 17.0103 & 4691 & 0.0035 & 7.3379 & 349 & 0.0202 & 1.0722 & 16 & 0.0519 \\
		\hline
		100 & 0.2 & - & - & - & - & - & - & 14.185 & 175 & 0.0787 \\
		\hline
	\end{tabular}
\hspace*{-4.2cm}
\section{Conclusión}
Es evidente la progresión exponencial del algoritmo, aun asi la aplicacion de cotas aun más costosas por nodo limita el numero de nodos y permite solucionar grafos más grandes y densos.
\paragraph{} El problema del cliqué es un problema estudiado en profundidaz como cualquier otro problema NP-Completo. Aun habiendo echo grandes avances en el campo que permiten en un tiempo mas o menos razonable gestionar grafos con muchas aristas, aun queda espacio por recorrer dado que la curva exponencial de su complejidad sigue limitando la resolución del problema incluso para ordenadores modernos.
\begin{thebibliography}{9}
\bibitem{car} Carraghan, R. \& Pardalos, P.M. (1990). \textit{An exact algorithm for the maximum clique problem.} Operations Research Letters, 9(6), 375–382.
\bibitem{tom} E. Tomita, T. Seki, \textit{An efficient branch-and-bound algorithm for finding
a maximum clique}, Lecture Notes in Computer Science 2631 (2003) 278-
289.
\bibitem{jonc} J. Konc \& D. Janežič, D. (2007). \textit{An improved branch and bound algorithm for the maximum clique problem.} MATCH - Communications in Mathematicaland in Computer Chemistry, 58, 569-590.
\end{thebibliography}
\appendix
\end{document} 