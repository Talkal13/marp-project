\documentclass{article}
\usepackage{mathtools}

\DeclarePairedDelimiter\abs{\lvert}{\rvert}%
\title{Practica de Metodos Algoritmicos de Resolución de Problemas}
\author{Pablo Vazquez Gomis  \\
	Universidad Complutense de Madrid \\
	}

\date{\today}
% Hint: \title{what ever}, \author{who care} and \date{when ever} could stand 
% before or after the \begin{document} command 
% BUT the \maketitle command MUST come AFTER the \begin{document} command! 
\begin{document}

\maketitle


\begin{abstract}
Practica opcional del primer cuatrimestre para la asignatura de Metodos Algorítmicos de Resolución de Problemas.
\end{abstract}

\section{Introducción}
\subsection{Descripción de la práctica}
Implementar o Java o en C++ un algoritmo, que dado un grafo dirigido, detecte si tiene o no
ciclos. En caso de ser acíclico, ha de listar sus vértices en orden topológico. Si hay más de uno
posible, los puede listar en cualquiera de ellos. En caso de ser cíclico, ha de listar sus componentes
fuertemente conexas (cada una es un conjunto de vértices). El algoritmo para esta segunda parte
puede verse en el Capítulo 22 del Cormen (2001, segunda edición).

\subsection{Implementación}

La práctica se puede dividir en 3 partes diferenciables:
\begin{enumerate}
\item \label{1} Comprobar si un grafo es cíclico o no
\item \label{2} los vértices en orden topológico
\item Listar los componentes fuertemente conexos 
\end{enumerate}

\paragraph{}Dado que en al ordenar los vértices de el punto 2 uno de los casos base devuelve error si el
grafo es cíclico y su complejidad esta en orden $ O(\abs{V} + \abs{E}) $ utilizaremos esa funcion para comprobar el punto 1.

\paragraph{}En caso positivo, habremos encontrado una solución al punto 1 en orden $ O(\abs{V} + \abs{E}) $. En caso negativo, tendremos la solución al punto 2
en el mismo orden.

\paragraph{}Para resolver el punto 3 utilizamos el algoritmo de Tarjan\cite{tarjan} para encontrar componentes fuertemente conexas,
el cual resuelve el problema utilizando \textit{Depth First Search (DFS)} con una complejidad de $ O(\abs{V} + \abs{E}) $.

\section{Ordenación topológica}

\section{Conclusions}\label{conclusions}


\begin{thebibliography}{9}
\bibitem{tarjan} Robert E. Tarjan. \textit{Depth-First Search and Linear Graph Algorithms}.
Stanford University. Stanford, California.
\end{thebibliography}

\end{document}